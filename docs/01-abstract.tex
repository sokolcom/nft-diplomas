\section*{РЕФЕРАТ}
\addcontentsline{toc}{section}{РЕФЕРАТ}


Расчетно-пояснительная записка к выпускной квалификационной работе <<Метод создания уникальных сертификатов, подтверждающих окончание учебного заведения, на основе технологии невзаимозаменяемых токенов>> содержит \pageref{LastPage} страниц, 4 части, \totalfigures\ рисунков, \totaltables\ таблиц, 33 источника.

Ключевые слова: блокчейн-сеть, невзаимозаменяемые токены, консенсус сети, смарт-контракты, документы об образовании.

Объект разработки -- метод создания сертификатов, подтверждающих окончание учебного заведения.

Цель работы – разработка и программная реализация метода создания уникальных сертификатов, подтверждающих окончание учебного заведения, с помощью блокчейн-сети, в которой содержаться невзаимозаменяемые токены.

Область применения -- учебные заведения и образовательные онлайн"=платформы, а также кадровые агентства, проверяющие резюме соискателей.


В первой части работы приведен обзор и сравнение существующих методов решения задачи выдачи сертификатов об образовании, средства разработки блокчейн-сети, а также алгоритмы консенсуса сети.

Во второй части описан процесс разработки метода создания уникальных сертификатов, подтверждающих окончание учебного заведения, с помощью блокчейн-сети и невзаимозаменяемых токенов.

В третьей части определены средства разработки, примененные при программной реализации метода, и приведена структура разработанного ПО.

В четвертой части проведены исследования доступности сертификата пользователям и зависимости времени финализации транзакций от числа узлов сети.


Поставленная цель была достигнута: был разработан и реализован метод создания уникальных сертификатов, подтверждающих окончание учебного заведения. Были рассмотрены преимущества и недостатки разработанного метода и предложены пути дальнейшего развития.

\pagebreak
