\section*{ЗАКЛЮЧЕНИЕ}
\addcontentsline{toc}{section}{ЗАКЛЮЧЕНИЕ}


В результате выполнения выпускной квалификационной работы были решены поставленные задачи:
\begin{itemize}[leftmargin=1.6\parindent]
	\item[---] проанализированы существующие способы выдачи сертификатов об окончании учебного заведения;
	\item[---] разработана блокчейн-сеть, в которой реализован метод выдачи уникальных сертификатов;
	\item[---] программно реализован разработанный метод;
	\item[---] исследована зависимость времени финализации транзакции от количества участников сети, а также произведено сравнение с аналогами; разработанное программное обеспечение полностью соответствует требованиям технического задания.
\end{itemize}

Был спроектирован и реализован метод создания невзаимозаменяемых токенов в блокчейн"=сети, с помощью которых можно подтвердить окончание учебного заведения.

Достоинства разработанного программного обеспечения по сравнению существующими аналогами:
\begin{itemize}[leftmargin=1.6\parindent]
	\item[---] защита от износа и подделывания выданных документов об образовании;
	\item[---] возможность предоставить электронное подтверждение подлинности документа, не оформляя бюрократических запросов;
	\item[---] неограниченное количество участников сети.
\end{itemize}

Основными недостатками разработанного программного обеспечения являются:
\begin{itemize}[leftmargin=1.6\parindent]
	\item[---] большие вычислительные мощности, требуемые от машины;
	\item[---] в случае если документ, хранящийся по ссылке в сертификате, скомпрометирован, тогда соответствующий сертификат тоже будет ложный.
\end{itemize}


Можно выделить следующие пути дальнейшего развития разработанной системы:
\begin{itemize}[leftmargin=1.6\parindent]
	\item[---] переписывание кода смарт"=контрактов на Rust;
	\item[---] произвести внешний аудит разработанных смарт"=контрактов;
	\item[---] развить блокчейн-сеть до парачейна Polkadot \cite{parachain}.
\end{itemize}

\pagebreak