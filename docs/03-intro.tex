\section*{ВВЕДЕНИЕ}
\addcontentsline{toc}{section}{ВВЕДЕНИЕ}


Документы об образовании играют важную роль в жизни каждого человека. Они необходимы для трудоустройства, повышения квалификации и т.п. В связи с этим встал вопрос о сохранности и подлинности данных документов. Документы об окончании государственных учебных заведений подвержены износу с течением времени; дипломы, выдаваемые образовательными онлайн-платформами, можно подделать в редакторе. Чтобы проверить подлинность документов об образовании необходимо сформировать запрос в государственный реестр или в организацию, выдавшую данный документ -- на это все необходимо дополнительное время.

Использование блокчейн-сети с невзаимозаменяемыми токенами для подтверждения окончание учебного заведения позволит авторизованным пользователям (учебным организациям) вместе с классическими документами выпускать специальный сертификат в сети, подтверждающий классические документ. Любой участник сети сможет убедиться в подлинности выпущенного сертификата, а с течением времени невозможно будет изменить или подделать данные сертификата, не нарушив состояние блокчейн-сети \cite{ru-bchain1}. Таким образом, использование данной технологии позволяет решить задачи, такие как утрата, подделывание и износ документов.


Целью работы является разработка и программная реализация метода создания невзаимозаменяемых токенов в блокчейн-сети, с помощью которых можно подтвердить окончание учебного заведения. В рамках работы решаются следующие задачи:
\begin{itemize}[leftmargin=1.6\parindent]
	\item[---] проанализировать существующие способы выдачи сертификатов об окончании учебного заведения;
	\item[---] разработать блокчейн-сеть, в которой реализован метод выдачи уникальных сертификатов;
	\item[---] программно реализовать разработанный метод;
	\item[---] исследовать зависимость времени финализации транзакции от количества участников сети, а также произвести сравнение с аналогами.
\end{itemize}



\pagebreak